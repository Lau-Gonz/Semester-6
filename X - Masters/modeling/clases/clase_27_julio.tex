\documentclass[a4paper,12pt]{article}


\usepackage[utf8]{inputenc}
\usepackage[spanish]{babel}
\usepackage{microtype}
\usepackage{graphicx}
\usepackage{wrapfig}
\usepackage{amssymb}
\usepackage{amsmath}
\usepackage{float}
\usepackage{hyperref}
\usepackage{listings}

\usepackage{import}
\usepackage{pdfpages}
\usepackage{transparent}
\usepackage{xcolor}
\usepackage[framemethod=default]{mdframed}

\definecolor{blue}{RGB}{61, 160, 250}

\author{Dave Alsina}
\title{{\color{blue}Class Notes}}



\newcommand{\incfig}[2][1]{%
        \def\svgwidth{#1\columnwidth}
        \import{./figures/}{#2.pdf_tex}
}

\pdfsuppresswarningpagegroup=1

\begin{document}
    \maketitle

    \begin{mdframed}{{\color{blue}Nota: }}
        Remember this is in english.
        In exams you can turn to spanish.
        Workshops in english.

        \begin{itemize}
            \item 2 credits theory (wendsday - theory)
            \item 1 parctice        (friday - practice)
        \end{itemize}

        work with matlab.
        1h and a half, for each part, theory and practice.

        This course will be theory driven, but in the road we will 
        use some applications.

        For the practical part of the exam we can use internet.
        For the theory we only can bring one sheet of paper.
    \end{mdframed}


    \section{What's a Dynamical sys?}
    
    \begin{itemize}
        \item System: It's a collection of parts, which act as a whole.
        \item Dynamical: Phenomena that are constatly evolving.
    \end{itemize}

    We will deal with:

    \begin{itemize}
        \item Difference equations.
        \item Ordinay Differential equations.
        \item Stochastic processes. $\rightarrow$ Stochastic differential equations.
    \end{itemize}


    \subsubsection{Difference equation (maps/iterative maps, they are functions)}

    Evolution is discrete (change occurs in discrete time).
    You can turn some continuos variable into discrete by modifiying the sampling
    period (i look my bank account once a month).
    
    The most conventional way to model discrete behaviour is some sort of 
    step wise funcion.

    $n \geq 1$, variables, and we have eqns that relate values at a certain 
    time to the adjacent values of the variables.\\

    \textit{Ex:} 

    \begin{align*}
        X_{k+1} &= X_{k} + b, b \in \mathbb{R}\\
                &\textit{The notation we will use, to emphasise that it's func of time:}\\
        x(k+1) &= x(k)  + b, b \in \mathbb{R}, k \in \mathbb{Z}^{*}\\
    \end{align*}

    Notice that $K$ is a discrete set. 
    If we talk about solutions of $x(k+1) = x(k)  + b, b \in \mathbb{R}$, we 
    mean functions, but particularly: \textbf{Sequences} of numbers which 
    satisfy the difference eqn.

    A solution to $x(k+1) = x(k)  + b, b \in \mathbb{R}$ is $0, b, 2b, 3b, ...$
    Another solution is for instance $1, b + 1, 2b + 1, ...$. To get a solution
    we need a particular point.


    \section{Differential eqs:}

    Evolution is continuos.
    (In general) they relate derivatives of $n\geq 1$ variables to their present
    values.

    \begin{align*}
        \dot{x} = ax(t), t \in \mathbb{R}, \mathbb{R}^{+}, \mathbb{R}^{*}
    \end{align*}

    \begin{mdframed}{{\color{blue}Nota: }}
        We can take derivaties relative to some complex variable, 
        but we won't do that here.
    \end{mdframed}

    A solution to the above equation, is a continuos function, a function of
    a continous set.
    A particular solution would be: $x(t) = 0$, $x(t) = c\cdot e^{at}$.\\

    \begin{mdframed}{{\color{blue}Nota: }}
        The choice of which type of equation to use for model some problem, Difference
        or Differential, depends on your will. Difference eqn. are easier to simulate,
        Differential eqn. are harder to simulate and usually have error, so need 
        some sort of aproximation. On maths Differential eqns are way easier than 
        Difference ones.
    \end{mdframed}


    \section{Multivariable systems}
    
        More than one variable is generally necesary to model behaviour of 
        systems with multiple parts.

        To relate variables Linear Algebra is used many times (vector
        representations of the variables), other times schematic diagrams
        are also used (only for intuition).\\

        \begin{mdframed}{{\color{blue}Nota: }}
            All models are wrong, but some are better than others.
        \end{mdframed}

        \textit{Ex:}

        \begin{enumerate}
            \item \textbf{Geometric growth:} 

            \begin{align*}
                x(k + 1) =  ax(k), k \in  \mathbb{N}
            \end{align*}
        \end{enumerate}

        A solution is $0, 0, ...$\\
        Another solution is: $1, a, a^{2}, a^3, ...$

        \begin{itemize}
            \item if $a \geq 1$, $x(k)$ grows or decays to $-\infty, \infty$
            \item if $0 < a < 1$, decays to zero.
            \item if $0 > a > -1$, it osilates around zero, while decaying.
        \end{itemize}

        \textit{Another example:}

        $k$: month. $x(k)$: number of pairs of bunnies.
        $x(k + 2) = x(k + 1) + x(k)$.\\

        \textit{Cohort population model example:} Population divide into age 
        groups od equal span (5 years). $x_{0}, x_{1}, x_{2}, ..., x_{n}$.
        $x_{0}$ represents the newborn individuals.\\
        
        if timestep = age span:\\
        certain death after 5(n+1) years, $x_{i+1}(k+1) = x_{i}(k)$.\\

        if there is a survival rate $\beta_{i}$:\\
        $x_{i+1}(k+1) = \beta_{i} x_{i}(k)$\\

        \textit{Another example:}
        Equations of force. $F = ma = m \cfrac{d^2 x}{dt^2}$


        \textit{Another example: Lotka-Volterra model predator-prey}.\\
        $N_{1}(t)$ population density of a prey at time t.\\
        $N_{2}(t)$ population density of a predator at time t.\\

        \begin{equation}
            \textit{Some derivarive system I wasn't able to copy.}
        \end{equation}

        \section{Stages of modeling}

            \begin{enumerate}
                \item Representation of the phenomena.
                \item Generation analysis/approximation of solutions.
                \item Exploration of structural relations. (describe the 
                    behaviour).
                \item Control, Predictions.
            \end{enumerate}

    

\end{document}




