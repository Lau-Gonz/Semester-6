\documentclass[a4paper,12pt]{article}


\usepackage[utf8]{inputenc}
\usepackage[spanish]{babel}
\usepackage{microtype}
\usepackage{graphicx}
\usepackage{wrapfig}
\usepackage{amssymb}
\usepackage{amsmath}
\usepackage{float}
\usepackage{hyperref}
\usepackage{listings}

\usepackage{import}
\usepackage{pdfpages}
\usepackage{transparent}
\usepackage{xcolor}
\usepackage[framemethod=default]{mdframed}

\definecolor{blue}{RGB}{61, 160, 250}

\author{Dave Alsina}
\title{{\color{blue}Notas Clase}}



\newcommand{\incfig}[2][1]{%
        \def\svgwidth{#1\columnwidth}
        \import{./figures/}{#2.pdf_tex}
}

\pdfsuppresswarningpagegroup=1

\begin{document}
        \maketitle

        \begin{mdframed}{{\color{blue}Nota: }}
        
        \end{mdframed}

        \section{Operacion}
        sea X un conjungo una operación (binaria)
        es una función $f : X \times X \rightarrow X$, se denota 
        $F(x,y) = x * y = x \cdot y$.


        \subsubsection{$\mathbb{R}^{3} \times \mathbb{R}^{3} \times \mathbb{R}^3$}
4
        Se llama producto mixto.
        $(u, v, w) \mapsto (u \land v)\cdot w$, $\land$ es producto vectorial.


        \subsection{Propiedades}

            \subsubsection{Cerradura}
            una operacion es necesariamente cerrada, $\forall x, y \in X$, 
            $x \cdot y \in X$.

            unos ejemplos de operación con cerradura:

            \begin{itemize}
                \item op. numerica.
                \item op. matrices.
                \item op. vectorial.
            \end{itemize}

        \begin{mdframed}{{\color{blue}Nota: }}
            Si X es finito podemos definir F a través de una tabla.
            por ejemplo una tabla de verdad.
        \end{mdframed}

        \subsubsection{Asociatividad}

            Una operación es asociativa si $\forall x, y, x \in  X$,
            $(x*y)*z = x*(y*z)$. Un contraejemplo:

            \begin{equation}
                e_{1} \cdot e_{1} \cdot e_{3} = e_{1} \cdot (e_{1} \cdot e_{3})
                \neq   (e_{1} \cdot e_{2}) \cdot e_{3} = 0 \cdot e_{2}
            \end{equation}

        \subsubsection{Conmutativa}

            si $\forall x, y, z \in X$, se tiene que $x * y = y * x$.

            Pueden existir elementos especiales:

            \begin{itemize}
                \item \it{Neuto:} $\exists e \in X$ tal que $\forall x \in X$,
                    $e * x = x$ and $x * e = x$.

                \item \it{inverso:} $\forall x \in X$, $\exists y \in X$:
                    $x * y = e = y * x$, el inverso me lleva al neutro.
            \end{itemize}

            \it{¿pueden existir múltiples neutros en una operación?}:
            
            \it{Dem:} Supongamos que $e, e' \in X$ son elementos neutros.

            porque $e' = e * e' = e$, porque $e'$ es neutro, nos lleva a que 
            son iguales ya que ambos dan neutro.
            
        \section{Clase de congruencia}

        Sea $m \in \mathbb{N}, m \geq 2$, Dos enteros $a,b$ son congrentes 
        módulo m.

        sii:

        \begin{equation}
            m|(x-y) \iff x - y = mt, t \in Z
        \end{equation}

        esta es relacion de equivalencia en $\mathbb{Z}$ se denota $x \equiv y (mod \;m)$.

        \begin{equation}
            [x] = {y \in Z: x \equiv y (mod \; n)}
        \end{equation}


        \subsubsection{Division euclidea:}

        si $a, b \in  Z, b \neq 0, \exists q, r \in Z$. t.q 
        $0 \leq z |b|$ y $a = bq + r$, $r$ de residuo

        Si $x = mq + r$ entonces: $[x] = [r]$.
        porque 

        \begin{equation}
            x - r = mq \iff m|(x-r) \iff x \equiv r (mod \; m)
        \end{equation}


        Existen tantas clases de congruencia cuantos residuos en la division 
        euclidea por m.

        $\mathbb{Z}_{m} = \{ [0], [1], ..., [m-1]\}$

        Queremos definir operaciones sobre el conjunto $\mathbb{Z}_{m}$. 
        Empecemos entonces por:

        Definimos:

        \begin{equation}
            [x]_{m} + [y]_{m} = [x + y]_{m}
        \end{equation}

        \begin{equation}
            [x]_{m} \cdot [y]_{m} = [x \cdot y]_{m}
        \end{equation}

        Un ejemplo de lo anterior es: 
        $[11]_{12} + [4]_{12} = [15]_{12} = [3]_{12}$

        otro ejemplo: 
        $[3]_{4} \cdot [3]_{4} = [9]_{4} = [1]_{4}$

        Uno complicado:
        $[2]_{4} \cdot [2]_{4} = [0]_{4}$.


        Otro ejercicio:
        \begin{align}
            3^{2001} &\equiv 3 \cdot 3^{2000} (mod 10)\\
                     &\equiv 3 \cdot 3(3^{4})^{500} (mod 10)\\
                     &\equiv 3 \cdot 1^{500}\\
                     &\equiv 3
        \end{align}


        \underline{dem:} Queremos Demostrar que estas operaciones están bien 
        definidas, es decir:

        si $[x]_{m} = [x']_{m}$ y $[y]_{m} = [y']_{m}$.

        Luego:

        \begin{align}
            [x + y]_{m} &= [x' + y']_{m}\\
            [xy]_{m} &= [x'y']_{m}
        \end{align}

        Siendo:

        \begin{align*}
            [x]_m &= [x']_m\\
            x &\equiv x' &(mod \; n)
        \end{align*}

        equivalentemente $m|(x - x')$, es decir $\exists n$


        \begin{mdframed}{{\color{blue}Nota: }}
            \begin{itemize}
                \item Hacer ejercicios como los anteriores de operación de módulo.
                \item Hacer la demostración del caso de la multiplicación, demostrar que 
            está bien definida. Estudiar la demostración completa.

            \end{itemize}
        \end{mdframed}



\end{document}
